
\chapter{Introduction} % Main chapter title

\label{Introduction} % Change X to a consecutive number; for referencing this chapter elsewhere, use \ref{ChapterX}

\lhead{Chapter 1. \emph{Introduction}} % Change X to a consecutive number; this is for the header on each page - perhaps a shortened title

For a few years, Computing students at \ref{university} have been required to
submit their first and second year projects via an internal gitlab system. The
aim of this requirement is to encourage students to familiarise themselves with
version control systems early in their Computing careers, as well as providing
a consistent method for submitting source code. The use of version control also
benefits the markers, as it allows them to use the commit messages to get a better
understanding of the students' intentions. Their behaviour in using version control
can also be examined.

\section{Motivation}

The current utilisation of gitlab data is minimal -- primarily it provides a
convenient form of code submission for the students, with little intention of
using the data to gather more in depth information on the students' performance.
The gitlab repositories provide us with the potential to both aid markers
in delivering consistent and correct grades and feedback, as well as allowing
the students to analyse their own work in comparison with others, so that they
can reflect on their own implementation. It is the underutilisation of this
data that I aim to overcome with this project.

\section{Challenges}

The main challenges within this project include:

\begin{itemize}

\item Analysis of plagiarism detection techniques and modification of algorithms
to return a similarity measure

\item Design a graphical representation for similarity of code amongst a large
number of files

\item Build a useable interface for students and lab coordinators to access the
data analyses

\item Improve the learning capability of students by allowing reflection on their
own performance when compared to their colleagues.

\end{itemize}