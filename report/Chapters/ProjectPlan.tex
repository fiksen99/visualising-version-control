
\chapter{Project Plan} % Main chapter title

\label{ProjectPlan} % Change X to a consecutive number; for referencing this chapter elsewhere, use \ref{ChapterX}

\lhead{Chapter 3. \emph{Project Plan}} % Change X to a consecutive number; this is for the header on each page - perhaps a shortened title

\section{Accomplished}

Much research has gone into the techniques used throughout academia to detect 
code similarity (see \ref{Background}). A number of approaches have been considered
and a particularly interesting approach has been chosen to focus on for the 
implementation of this project. To apply the given algorithms, I need access to
a set of code bases to compare, which has been provided by Mark Wheelhouse, along
with the grades assigned to the code.

\section{Immediate Priorities}

Having studied a number of research articles in the area of plagiarism detection,
the next step is to begin working on a prototype implementation of my chosen
method (semantic/structural analysis). With access to the second year pintos labs,
written in C, and the grades assigned to the students, I can begin testing various
approaches and modifications to algorithms. The outcome of these experiments will
determine the next area to work on -- a decent enough result will allow me to
begin working on a user interface/graphical representation for the data, which
can be distributed to a number of students to test. Alternatively, it is probable
that more research will be needed, so delving further into more recent plagiarism
detection papers may be necessary.

This step is a lot more in depth than it seems. As discussed in the literature,
we need to build a lexer/parser for our language of choice to construct our
internal tree represenation. Tools to use here include flex~\cite{flex} and bison
\cite{bison} to build the tree, and then application of our chosen algorithm
to generate the results.

\section{Building the Interface}

Once a suitable algorithm has been designed, the next step is to design and 
implement the user interface of the application. At present, a gitlab plugin
seems the most appropriate method of achieving this, and, as stated, the
form in which the data is displayed will be investigated with feedback given by
potential users of the system.

\section{Additions to the System}

Depending on the progress of the project, it could be possible to add extra
features to the plugin. Previously touched upon ideas include integration with
testing systems and other comparisons between code bases. These ideas could also
be included in a contingency plan -- as the complexity of these features is much
smaller, if the similarity comparison were to fall through, a final product could
still be achieved with useful results. This will give an actual application to
present in the final stages alongside the analysis of problems with similarity
comparison.

\section{Timeline}
\begin{figure}[ht!]
\centering
\begin{tabular}{ | l | l | l| }
    	\hline
    	\textbf{Task} & \textbf{Effort/Days} & \textbf{Expected Completion Date/`w/b'} \\ \hline
    	\hline
    	Research potential algorithms & ongoing & ongoing \\ \hline
    	Implement prototype of algorithm & \_ & 10/03 \\ \hline
    	-- Build lexer parser & 4 & 17/02 \\ \hline
    	-- Implement algorithm & 4 & 24/02 \\ \hline
    	User interface & \_ & 24/03 \\ \hline
    	-- Design & 1 & 10/03 \\ \hline
    	-- Implementation & 2 & 24/03 \\ \hline
    	-- Feedback and implementation update & ongoing & ongoing \\ \hline
    	Improve algorithm & 10 & 14/04 \\ \hline
    	Additional features and improvements & 15 & 05/05 \\ \hline
    	Writeup & ongoing + 10 & 19/05 \\ \hline
\end{tabular}
\emph{assuming an effort of 1.5 days per week until w/b 31/03, 5 days per week afterwards}		
\end{figure}

This timeline allows plenty of time to complete all tasks, with plenty of room to
manouver if some deadlines are missed.