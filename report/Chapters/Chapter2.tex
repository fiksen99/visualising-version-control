\chapter{Background}

\label{Chapter2}

\lhead{Chapter 2. \emph{Background}}

%----------------------------------------------------------------------------------------
%	Initial Approach
%----------------------------------------------------------------------------------------

\section{Initial Approach}

The first problem encountered immediately in my project was in specifying
specifically what problem I wanted to solve, and which direction I should take
my project in. The initial proposition by my supervisor was very broad, 
enabling me to look into a wide variety of goals. In particular, I saw two separate areas
of focus -- there was a clear divide between looking \emph{within} individual git 
projects and comparing statistics \emph{between} multiple projects, with both 
enabling interesting data to be gathered. 

The possibilities of looking within projects is quite exciting, as it could enable a solution
to a problem that appears fairly regularly in industry -- the handover problem \cite{handover}.
For example, if we were to provide information about certain files and even lines which are 
often modified in commits, it could highlight an area with a high potential for error.
Further to this, we could integrate the git commit analyses with a testing
framework, creating a tool to indicate good and bad coders -- those who add
tests, those who frequently break them etc.

There is also interesting potential in examining the data given between different
git projects. When we consider undergraduate work in earlier years, there are a
large number of courseworks which have the same specification, completed by 
individuals. This will allow us to directly compare statistics on these pieces
of work and gain some insight into the students' workflow, and even how this
compares to their grade received. Some areas of note in this area include comparing
areas of code modified, their start and end times, and the methodologies used
by undergraduates vs. postgraduates working on the same projects. The results
could easily be cross referenced against their grades to indicate successful
or poor approaches.
%-----------------------------------
%	SUBSECTION 1
%-----------------------------------
\subsection{Subsection 1}


%----------------------------------------------------------------------------------------
%	SECTION 2
%----------------------------------------------------------------------------------------

\section{Areas of Research}
