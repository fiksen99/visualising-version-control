
\chapter{Evaluation Plan} % Main chapter title

\label{EvaluationPlan} % Change X to a consecutive number; for referencing this chapter elsewhere, use \ref{ChapterX}

\lhead{Chapter 4. \emph{Evaluation Plan}}

The primary goal for this project is to develop a tool that can be used by lab
coordinators and students. The main feature of this tool will be to give a
graphical representation of the similarity of students' code. A key method in
testing the usefulness of this measure will be an analysis into the correlation
of marks with a high similarity measure. If the data shows a high similarity
between to pieces of work whilst the grades vary significantly, this would indicate
an error within the algorithm. 

This goal adds a large element of risk to the project -- research into code
similarity detection methods is still ongoing, with many conflicting methods. As
such, it is very possible that this goal is unreachable as part of this project;
for example if it is found that there is too much variation in the similarity
measure and no real clusters can be formed (even when code appears similar through
manual inspection), then there would be little use to this tool. This, however, is
still an interesting result, as we will spend a lot of time analysing how well
the algorithms presented perform, and uncover reasons why they fail for our
purposes and possible further areas of research which may yield more useful results.

Further to the success of the algorithm, it is also vital that an intuitive and
helpful interface is developed for users. The utility of the project relies on
lab coordinators and students being able to easily visualise the information
gathered. This will be evaluated in a continuous process, where various attempts
will be released to a sample students and feedback gathered. 

An added bonus to the project will be if there is time to expand the suite of 
tools made available. Proposed ideas such as further git behaviour analysis
or integration with testing would add more appropriate functionality. Implmenting
these ideas aren't a main focus of the project, but would be a nice addition,
their success judged in the same way as the similarity measure.

The final evaluation of the
project will come with the possibility of releasing to the undergraduate population.
If people believe the tool is good enough to use on their exercises, then the project
would be considered a great success.